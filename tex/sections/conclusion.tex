\subsection{Summary}
It has been a long and arduous journey to get here, but we have now sucessfully implemented a memory allocator (Milestone 1), enabled user-space virtual memory management and paging (Milestone 2), spawned new processes (Milestone 3), booted additional cores (Milestone 5), and have a mechanism to allow processes across cores to communicate (Milestones 4 and 6). In Milestone 7 we extended our system to have more specialized features such as networking, a more advanced capability management system, a shell, and a nameserver.  

\subsection{Outcomes}
The project provided an opportunity to improve both technical and non-technical skills for all group members.
\begin{itemize}
    \item \textbf{Understanding microkernel architecture}:
    We entered this class under the expectation that we'd be taught how the Linux kernel works, as it is currently the standard in most parts of industry and academia. It was a bit of a surprise to be shown a more modern system architecture, which moved away from the monolithic kernel. It was overwhelming at the beginning having to both grasp completely new concepts (which fly in the face of all the ones we're used to), \textit{and} implement them at the same time. Ultimately, we are incredibly appreciative to have been given the opportunity to explore modern kernel design, as it is likely to benefit us when we move into the industry or higher levels of academia. 
    \item \textbf{Organizing group work}:
    All of us had previously experienced group work in a co-op setting, so this was not really foreign to us. However, it was a comparatively smaller codebase that we were working on compared to what we had experienced during co-op, and oftentimes we had to write code on top of each others' in realtime as opposed to things being a bit more isolated. It is much easier to write code than to understand code, and we faced challenges keeping everyone up-to-date on the state of the system while Milestone tasks were being completed. We feel that working in such close conditions allowed us to greatly strengthen our code-comprehension abilities in a short amount of time.
    \item \textbf{Handling a large code base}: Jumping into a system like Barrelfish is initially very daunting due to its many layers and interacting subsystems. It takes skill to determine which parts of the code base are relevant for a particular task and to ensure that subsystems continue to behave as expected by other parts of the code. To make this easier, we ensured that new additions to the code base were as isolated as possible, and used techniques like modular debug messages to keep the logs from becoming overwhelming.
    \item \textbf{Underdefined problems}: Some of the tasks did not describe a particular implementation and allowed us design freedom. Many of these decisions would affect us in later Milestones which exemplified the importance of careful planning. On the other hand, we also learned that spending too much time deliberating on a design can hinder progress. It was beneficial to design with flexibility in mind, so that we could more easily change our architecture in the future if necessary.
\end{itemize}
